\documentclass{acm_proc_article-sp}
\begin{document}
\title{Algorithm and Data Structure Coursework: \\PCA Features for
R-tree Based Similar Image Search}
\subtitle{}
%
\numberofauthors{2} %  in this sample file, there are a *total*
% of EIGHT authors. SIX appear on the 'first-page' (for formatting
% reasons) and the remaining two appear in the \additionalauthors section.
%
\author{
% You can go ahead and credit any number of authors here,
% e.g. one 'row of three' or two rows (consisting of one row of three
% and a second row of one, two or three).
%
% The command \alignauthor (no curly braces needed) should
% precede each author name, affiliation/snail-mail address and
% e-mail address. Additionally, tag each line of
% affiliation/address with \affaddr, and tag the
% e-mail address with \email.
%
\alignauthor
Qiwei Feng\\
       \affaddr{2011011250, IIIS-10}\\
       \affaddr{Tsinghua University}\\
       \email{gdfqw93@163.com}
\alignauthor
Pufan He\\
       \affaddr{2011011307, IIIS-10}\\
       \affaddr{Tsinghua University}\\
       \email{hpfdf@126.com}
}
\date{26 May 2015}

\maketitle
\begin{abstract}
This paper provides a sample of a \LaTeX\ document.
\end{abstract}

\keywords{R-Tree, Similar Image, PCA, K-Means}

\section{Introduction}

Github address:......

\subsection{Image Search}

\subsection{Data Structures}

\subsection{Low Level Features}

\subsection{PCA}

\subsection{K-Means}

\section{Data}

\section{Feature Finding}

\section{R-Tree}
We use the ``rtree alternative package'' implementation of R-tree.
The wrapper \texttt{src/a.cpp} calls methods of provided R-tree class.

\section{Experiments}

\section{Conclusion}

\begin{table}
\centering
\caption{Frequency of Special Characters}
\begin{tabular}{|c|c|l|} \hline
Non-English or Math&Frequency&Comments\\ \hline
\O & 1 in 1,000& For Swedish names\\ \hline
$\pi$ & 1 in 5& Common in math\\ \hline
\$ & 4 in 5 & Used in business\\ \hline
$\Psi^2_1$ & 1 in 40,000& Unexplained usage\\
\hline\end{tabular}
\end{table}

\begin{table*}
\centering
\caption{Some Typical Commands}
\begin{tabular}{|c|c|l|} \hline
Command&A Number&Comments\\ \hline
\texttt{{\char'134}alignauthor} & 100& Author alignment\\ \hline
\texttt{{\char'134}numberofauthors}& 200& Author enumeration\\ \hline
\texttt{{\char'134}table}& 300 & For tables\\ \hline
\texttt{{\char'134}table*}& 400& For wider tables\\ \hline\end{tabular}
\end{table*}
% end the environment with {table*}, NOTE not {table}!


\bibliographystyle{abbrv}
\bibliography{sigproc} 

\balancecolumns

\end{document}
