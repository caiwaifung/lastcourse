\documentclass{acm_proc_article-sp}
\usepackage{tikz}
\usetikzlibrary{plotmarks}
\begin{document}
\title{Algorithm and Data Structure Coursework: \\K-Means Feature for
Image Retrieval}
\subtitle{}
%
\numberofauthors{2} %  in this sample file, there are a *total*
% of EIGHT authors. SIX appear on the 'first-page' (for formatting
% reasons) and the remaining two appear in the \additionalauthors section.
%
\author{\alignauthor
Qiwei Feng\\
       \affaddr{2011011250, IIIS-10}\\
       \affaddr{Tsinghua University}\\
       \email{gdfqw93@163.com}
\alignauthor
Pufan He\\
       \affaddr{2011011307, IIIS-10}\\
       \affaddr{Tsinghua University}\\
       \email{hpfdf@126.com}
}
\date{16 June 2015}

\maketitle
\begin{abstract}
        This project implements a similar image search algorithm (image
        retrieval) based on multiclass classification and K-Means feature. Our
        training phase includes image resizing, image patch extraction, patch
        sampling, PCA whitening, K-Means for patches, feature extraction and
        multiclass SVM. We use 218 dimension K-Means and RGB, HSV color moment.
        The training phase takes no greater than one hour in time, 8GB in memory.
        Finally we obtained 69.49\% accuracy on test data classification.

We made our work open, and the full project codes can be found at
\texttt{https://github.com/caiwaifung/lastcourse}.
\end{abstract}

\keywords{Image Retrieval, Image Classification, SVM, Whitening, K-Means}

%------------------------------------------------------------------------%
\section{Introduction}
% say what we want to do, want we did, how well we could make
% briefly discuss how we did: the main part is in "Implementation" section

%------------------------------------------------------------------------%
\section{Implementation}
% say how we implementation the system

\subsection{Patch Extracting and Sampling}
\subsubsection{Whitening}

\subsection{K-Means Clustering}

\subsection{Feature Extracting}

\subsection{Multiclass SVM}

%------------------------------------------------------------------------%
\section{Experiments}

\subsection{Data Set}

\subsection{Without Whitening}

\subsection{With Whitening}

\subsection{Final Test}

%------------------------------------------------------------------------%
\section{Conclusion}
% say some nonsense

%------------------------------------------------------------------------%
\nocite{*}
\bibliographystyle{abbrv}
\bibliography{sigproc} 

\balancecolumns

\end{document}
