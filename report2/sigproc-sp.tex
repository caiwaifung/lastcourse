\documentclass{acm_proc_article-sp}
\usepackage{tikz}
\usepackage{url}
\usetikzlibrary{plotmarks}
\begin{document}
\title{Algorithm and Data Structure Coursework: \\K-Means Feature for
Image Retrieval}
\subtitle{}
%
\numberofauthors{2} %  in this sample file, there are a *total*
% of EIGHT authors. SIX appear on the 'first-page' (for formatting
% reasons) and the remaining two appear in the \additionalauthors section.
%
\author{\alignauthor
Qiwei Feng\\
       \affaddr{2011011250, IIIS-10}\\
       \affaddr{Tsinghua University}\\
       \email{gdfqw93@163.com}
\alignauthor
Pufan He\\
       \affaddr{2011011307, IIIS-10}\\
       \affaddr{Tsinghua University}\\
       \email{hpfdf@126.com}
}
\date{16 June 2015}

\maketitle
\begin{abstract}
        This project implements a similar image search algorithm (image
        retrieval) based on multiclass classification and K-Means feature. Our
        training phase includes image resizing, image patch extraction, patch
        sampling, PCA whitening, K-Means for patches, feature extraction and
        multiclass SVM. We use 218 dimension K-Means and RGB, HSV color moment.
        The training phase takes no greater than one hour in time, 8GB in memory.
        Finally we obtained 69.49\% accuracy on test data classification.

We have made our work open, and the full project codes can be found at
\texttt{https://github.com/caiwaifung/lastcourse}.
\end{abstract}

\keywords{Image Retrieval, Image Classification, SVM, Whitening, K-Means}

%------------------------------------------------------------------------%
\section{Introduction}
% say what we want to do, want we did, how well we could make
% briefly discuss how we did: the main part is in "Implementation" section

%------------------------------------------------------------------------%
\section{Implementation}
% say how we implementation the system

\subsection{Patch Extracting and Sampling}
\subsubsection{Whitening}

\subsection{K-Means Clustering}

\subsection{Feature Extracting}

\subsection{Multiclass SVM}

%------------------------------------------------------------------------%
\section{Experiments}

\subsection{Data Set}

\subsection{Without Whitening}

\subsection{With Whitening}

\subsection{Final Test}

Please see \texttt{a.html} under \texttt{result.zip} for a more detailed demo.

%------------------------------------------------------------------------%
\section{Conclusion and Future Work}
We have implemented a full workflow of image retrieval problem. Our program is
integrated in one matlab module, and almost all parameters can be adjusted. We
implemented our own K-Means algorithm, and visualized our K-Means result on
patch clustering into $K$ images. The centroids we got prove to meet clear
patterns, which is similar to other successful convolutionary computer vision
systems. Our training phase and feature extraction process are highly
optimized, so that training 5000 images takes less than one hour, and
classifying 2000 test images takes less than one minutes. Our final accuracy
rate 69.49\% is also competitive in all current image 10-classification
algorithms using the same level of computing resources. The closest retrieval
demo brings very reasonable results as well.

However there are still many wrong predictions that are trivial for human. If
we use deeper machine learning model such as CNN, we may further improve our
accuracy. We can also generate small noises, random rotation, flipping, and
many other tricks to enrich the dataset for larger machine learning framework.
So if time permitting, we will try those algorithms.

%------------------------------------------------------------------------%
\nocite{*}
\bibliographystyle{abbrv}
\bibliography{sigproc} 

\balancecolumns

\end{document}
